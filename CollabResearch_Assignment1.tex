\documentclass[]{article}
\usepackage{lmodern}
\usepackage{amssymb,amsmath}
\usepackage{ifxetex,ifluatex}
\usepackage{fixltx2e} % provides \textsubscript
\ifnum 0\ifxetex 1\fi\ifluatex 1\fi=0 % if pdftex
  \usepackage[T1]{fontenc}
  \usepackage[utf8]{inputenc}
  \usepackage{eurosym}
\else % if luatex or xelatex
  \ifxetex
    \usepackage{mathspec}
    \usepackage{xltxtra,xunicode}
  \else
    \usepackage{fontspec}
  \fi
  \defaultfontfeatures{Mapping=tex-text,Scale=MatchLowercase}
  \newcommand{\euro}{€}
\fi
% use upquote if available, for straight quotes in verbatim environments
\IfFileExists{upquote.sty}{\usepackage{upquote}}{}
% use microtype if available
\IfFileExists{microtype.sty}{%
\usepackage{microtype}
\UseMicrotypeSet[protrusion]{basicmath} % disable protrusion for tt fonts
}{}
\usepackage[margin=1in]{geometry}
\ifxetex
  \usepackage[setpagesize=false, % page size defined by xetex
              unicode=false, % unicode breaks when used with xetex
              xetex]{hyperref}
\else
  \usepackage[unicode=true]{hyperref}
\fi
\hypersetup{breaklinks=true,
            bookmarks=true,
            pdfauthor={Benedikt Abendroth and Katrin Heger},
            pdftitle={Research Proposal},
            colorlinks=true,
            citecolor=blue,
            urlcolor=blue,
            linkcolor=magenta,
            pdfborder={0 0 0}}
\urlstyle{same}  % don't use monospace font for urls
\setlength{\parindent}{0pt}
\setlength{\parskip}{6pt plus 2pt minus 1pt}
\setlength{\emergencystretch}{3em}  % prevent overfull lines
\setcounter{secnumdepth}{0}

%%% Use protect on footnotes to avoid problems with footnotes in titles
\let\rmarkdownfootnote\footnote%
\def\footnote{\protect\rmarkdownfootnote}

%%% Change title format to be more compact
\usepackage{titling}

% Create subtitle command for use in maketitle
\newcommand{\subtitle}[1]{
  \posttitle{
    \begin{center}\large#1\end{center}
    }
}

\setlength{\droptitle}{-2em}
  \title{Research Proposal}
  \pretitle{\vspace{\droptitle}\centering\huge}
  \posttitle{\par}
  \author{Benedikt Abendroth and Katrin Heger}
  \preauthor{\centering\large\emph}
  \postauthor{\par}
  \predate{\centering\large\emph}
  \postdate{\par}
  \date{October 23, 2015}



\begin{document}

\maketitle


\section{Introduction And Research
Question}\label{introduction-and-research-question}

After the end of the Cold War there was optimism that the number of
conflicts would decrease. But this hope was dashed with the emergence of
numerous intrastate conflicts in Europe, Africa, and Asia. Although the
incidences of conflict have declined over the last decade, the world
seems far from being in peace with conflicts raging in Syria, Ukraine,
Yemen, South Sudan and the DR Congo. Not only are these internal violent
armed conflicts associated with high costs in terms of human lives, they
also plummet states into economic depression and spill over to
neighboring countries, thereby negatively affecting the economy and
stability of entire regions. The international community finds itself in
the position of deciding whether it should merely stand idly by or
actively intervene into conflicts, may it be for humanitarian or
strategic reasons. A growing reluctance to engage militarily in states
at war is partly due to a decreasing willingness to risk one's own
soldiers' lives in conflicts abroad and partly due to the failure rate
of previous military interventions.

In this context, delivering weapons to conflict zones has become a
convenient solution for many Western states that constitutes a middle
ground between being a passive bystander and sending troops. In Syria,
the US provided weapons to the rebels fighting the autocratic Assad
regime and in Ukraine, the pro-government forces have received arms in
order to fight the separatist movement. This idea of helping warring
parties help themselves is not new. During the Cold War, the US and the
USSR frequently lent ``support of an indirect nature'' (Henderson 2013,
643) to government and opposition groups in so-called proxy wars. The
delivery of arms to war zones like Nicaragua or Laos was conducted
covertly, with the intervenors being aware of both the illegality and
the questionable morality of such weapons transfers.

While there are all sorts of legal and moral deliberations dominating
the debate on interventions in policy circles as well as in academia,
the question of whether this form of intervention actually works is
addressed less systematically. This paper tries to fill this gap by
investigating the relationship between the delivery of weapons as one
form of international intervention and its success in ending civil war.

\section{Literature Review}\label{literature-review}

In the literature, several options for outside actors to intervene in
conflicts have been explored theoretically. According to Snyder and
Jervis (1999), conflict situations can be analyzed in the framework of a
security dilemma. A ``security dilemma is a situation in which each
party's efforts to increase its own security reduce the security
altogether'' (Snyder/Jervis 1999, 15). The idea of a security dilemma
stems from the philosophy of Thomas Hobbes, who laid out the state of
nature as a state of war. He deduces his state of nature from the
condition that people are similar in their mental and physical
capacities and thus no one is able to dominate the other. Humans are
further inherently egoistic and rational in their striving to achieve
their own goals. Under conditions of anarchy and resource scarcity,
these attributes of human nature lead to a violent struggle for
survival. People feel threatened by others and engage in pre-emptive
attacks as a means of anticipatory defense, which ultimately results in
a spiral of violence, a ``war of all against all''. In the state of
nature, people lead a life that is ``solitary, poor, nasty, brutish, and
short'' (Hobbes 1651, Chapter 13). In order to escape this situation,
they voluntarily agree to appoint a sovereign government, the state, by
entering in a social contract. In exchange for giving up (some of) their
freedom, they are guaranteed peace and security (ibid).

While the security dilemma is primarily applied to describe the
relationship between states in international relations theory, some
scholars have adopted it as a tool to analyze civil wars
(Visser/Duyvesteyn 2014). According to this narrative, civil wars very
much resemble the Hobbesian state of nature, in which conditions of
anarchy due to lacking state capacities and competition over territory
and power might make attack the best form of defense when one feels
threatened. Within the framework of the security dilemma, ending a war
means breaking off the dilemma structure. One resolution to the civil
war security dilemma is one party achieving military victory and as a
result taking over the rule. Although a winner dominating the losers of
the war is not equal to Hobbesâ\euro{}™ idea of a sovereign to whom the
citizens of a state voluntarily render parts of their sovereignty, the
underlying mechanisms are the same. The winner will disarm his opponents
and unite all the power, thereby creating stability (Duffy Toft 2010).
Empirical evidence proves that civil wars ending in the military defeat
of one party are less likely to relapse into violence than conflicts
ending with negotiated settlements. This finding has prompted some
scholars to argue for a laissez-faire approach towards civil conflicts,
suggesting that it might be best to ``give war a chance'' (Duffy Toft
2010, 7) and let the parties fight until a clear winner emerges.
Applying this logic to foreign interventions, Snyder and Jervis (1999)
and Regan (1996) suggest for third-party interveners to change the
dilemma structure either by increasing the costs for the parties to
fight and reducing the costs of peace or by putting in place a hegemonic
power that imposes peace on the warring parties. While changing the
incentive structure can be achieved by negotiating peace agreements or
by implementing monitoring mechanisms like peacekeeping operations,
dominating power can be established by implementing outside rule or by
strengthening one side in the conflict in order to make it the dominant
group (Snyder/Jervis 1999, 27).

In an early attempt to quantify the relationship between foreign
intervention and civil wars, Regan (1996) investigated the conditions
for successful third-party intervention into intrastate wars. He found
that it is the type of the intervention strategy rather than the nature
of the conflict that determines the success of an intervention.
Distinguishing between two intervention strategies, military and
economic intervention, Regan established that a mix of the two
strategies rather than either type alone would be more effective.
However, he found that in general, conflicts experiencing external
interventions are less likely to terminate than those without. These
findings were confirmed in Regan's subsequent research that modeled the
success of an intervention as a function of (1) the strategic
environment in which the conflict is being waged; (2) the existence of a
humanitarian crisis associated with the conflict; (3) the number of
fatalities; and (4) the intensity of the conflict (Regan 2002).

The early 2000s saw a broad range of similar quantitative research,
establishing links between intervention and the duration of civil war as
well as the number of casualties. Lacina (2006) testifies that conflicts
with interventions result in significantly higher numbers of fatalities
and Cunningham (2010) evokes that the involvement of third parties
significantly prolongs civil wars. Elbadawi \& Sambanis (2000) modeled
in how far supporting the rebels or supporting the government makes a
difference for the duration of conflict. Taking into account the
strategic considerations rebels and governments make when faced with a
potential intervention, they find that the net effect of intervention on
the duration of conflict is negative. For example, Elbadawi \& Sambanis
discovered that interventions backing rebels against autocratic regimes
prolong conflicts because they strengthen rebellions that could have
been defeated easily otherwise.

Most of this research, however, failed to disentangle various forms of
interventions by lumping together military intervention on the one and
economic intervention on the other hand. This generic approach to
intervention opens up a gap for us to investigate the specific effects
of arms supply to one side of the conflict.

\section{Data Sources}\label{data-sources}

Using datasets from the University of Uppsala Conflict Database Program
(UCDP), we seek to examine the relationship between delivering lethal
military equipment, weapons, as one form of third-party intervention and
their potential to end violent conflicts. Specifically, we are pulling
our data from the Conflict Termination dataset (Kreutz 2010) and a UCDP
dataset on external support to parties in intrastate conflicts,
capturing various forms of third-party support (Hoegbladh et al. 2011).

There are obvious limitations to the data and it might be biased for two
reasons: First, data might be incomplete because weapons transfers might
happen covertly due to legal issues and strategic reasons. Second, data
on lethal aid might bear a selection bias since it is likely that only
countries that assumed they would be successful with an intervention
decide to intervene and are included in the sample (Regan 1996, 342).

\section{Methodology}\label{methodology}

We base our definition of conflict on the terminology introduced by UCDP
together with the Peace Research Institute Oslo (PRIO) (Themner and
Wallensteen 2011; Gleditsch et al. 2002). A `major armed conflict' or
`war' is characterized by at least 1,000 battle related, military or
civilian, deaths a year, while conflicts with 25 to 999 battle deaths a
year are categorized as `minor armed conflict'. They further distinguish
other forms of violence from conflicts by adding the component of an
organized effective violent opposition to the government, thereby
excluding genocides and similar forms of violence from their definition
of conflict.

Drawing from Regan (1996), we call our dependent variable success of
intervention, which is operationalized as the ``acessation of military
hostilities for a period lasting at least six months'' (Regan 1996,
343). Our independent variable is weapons delivery to one party in the
conflict with the aim of enabling it to defeat the other side.

In order to isolate the effect of arms transfers on the termination of
civil wars, we will have to control for conflict dynamics and external
factors. Conflict dynamics can be measured in terms of the type of
conflict, the military strength of the warring parties, the time period
before the intervention, and whether the supported party is the
government or the rebel group. External circumstances we might have to
control for is other strategies used simultaneously by third parties,
such as economic aid or weapons embargoes on the opposing faction.

\section*{References}\label{references}
\addcontentsline{toc}{section}{References}

\end{document}
